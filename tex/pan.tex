\begin{volumetitlepage}
	\volumetitle{Pan}
	\volumeheader{Pan}
\end{volumetitlepage}

\poemtitle{i}

\begin{artItem}
	Aleksandr Michajlovič Rodčenko, \begin{otherlanguage}{russian}%
		Белый круг%
	\end{otherlanguage}
\end{artItem}

\begin{poem}
	\begin{stanza}
		ma sono la stessa cosa\verseline
		sonno veglia\verseline
		vigilanza incoscienza\verseline
		logos panico\verseline
		e frontiera\verseline
		e il grembo incerto\verseline
		della terra
	\end{stanza}
\end{poem}

\clearpage

\poemtitle{ii}

\begin{artItem}
	Marcel Duchamp, \begin{otherlanguage}{french}%
		Rotorelief n\textsuperscript{o}3%
	\end{otherlanguage}
\end{artItem}

\begin{poem}
	\begin{stanza}
		è specchio il dio\verseline
		è solitario amore\verseline
		è sguardo eternamente\verseline
		volto al fluire
	\end{stanza}
\end{poem}

\clearpage

\poemtitle{iii}

\begin{artItem}
	Philippe Decrauzat, \begin{otherlanguage}{english}%
		BSBTE (Black Should Bleed To Edge)%
	\end{otherlanguage}
\end{artItem}

\begin{poem}
	\begin{stanza}
		mi vieni incontro sul limite\verseline
		della sera o del sonno\verseline
		mi mostri permeabile\verseline
		la frontiera\verseline
		giacché ti si addice\verseline
		stare di fronte
	\end{stanza}

	\begin{stanza}
		pure sono tuoi per intero\verseline
		terrore del meriggio\verseline
		strada isolata\verseline
		folgorazione non mediata\verseline
		spazio senza ragione
	\end{stanza}

	\begin{stanza}
		tu inflessibile\verseline
		lume superno\verseline
		generi creature\verseline
		pietrose\verseline
		che sanno sguardi\verseline
		di pietra
	\end{stanza}
\end{poem}

\clearpage

\poemtitle{iv}

\begin{poem}
	\begin{stanza}
		a cosa assomigliarti?
	\end{stanza}

	\begin{stanza}
		a una bestiola dei boschi —\verseline
		un volpino un leprotto che guizza tra l'erba\verseline
		una daina giovinetta dalla tenera gola —\verseline
		oppure a un agnello capriccioso\verseline
		ignaro ancora dei morsi?
	\end{stanza}

	\begin{stanza}
		ma sa l'agnello la rupe\verseline
		e la daina il signore radioso\verseline
		sanno il leprotto il volpino\verseline
		il silenzio al meriggio e i cani\verseline
		e l'inutile fuga?
	\end{stanza}
\end{poem}

\clearpage

\poemtitle{v}

\begin{artItem}
	Sabin Bălașa, \begin{otherlanguage}{romanian}%
		Compoziţie marină%
	\end{otherlanguage}
\end{artItem}

\begin{poem}
	\begin{stanza}
		tu inflessibile\verseline
		lume superno\verseline
		generi creature\verseline
		pietrose\verseline
		che sanno sguardi\verseline
		di pietra
	\end{stanza}
\end{poem}

\clearpage

\poemtitle{vi}

\begin{artItem}
	Max Ernst, \begin{otherlanguage}{german}%
		Antipoden Landschaft%
	\end{otherlanguage}
\end{artItem}

\begin{poem}
	\begin{stanza}
		arcade dai molti nomi\verseline
		tu che rechi la luce\verseline
		ma godi degli antri\verseline
		proteggi il nostro limite\verseline
		e salvaci dalla pazzia\verseline
		la peggiore la più umana\verseline
		tra le nostre paure
	\end{stanza}
\end{poem}
