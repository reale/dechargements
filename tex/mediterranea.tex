\begin{volumetitlepage}
	\volumetitle{Mediterranea}
	\volumeheader{Mediterranea}

	\bigskip\bigskip\bigskip
	\volumeepigraph{
			\begin{otherlanguage}{english}
                                Ay, they have deserted Greece, the Sirens.
			\end{otherlanguage}
	}
        \volumeattribution{Norman Douglas, Siren Land}
\end{volumetitlepage}

\poemtitle{parodo}

\begin{poem}
	\begin{stanza}
		Maestro
	\end{stanza}

	\begin{stanza}
		mi imponi ritorno\verseline
		alle coste scorte quegli anni e forse\verseline
		scordate
	\end{stanza}

	\begin{stanza}
		gridi richiami
	\end{stanza}

	\begin{stanza}
		ma fanciullezza è passata\verseline
		e che vale se fosse ignara\verseline
		meno di questi giorni maturi\verseline
		meno di questa efficienza\verseline
		fatta di cosa?
	\end{stanza}

	\begin{stanza}
		non sai se magari nulla più io so\verseline
		di allora\verseline
		se non voglia di ritorno
	\end{stanza}

	\begin{stanza}
		ma Itaca è fuori portata
	\end{stanza}

	\begin{stanza}
		allora a che gridi richiami? ancora?
	\end{stanza}

	\begin{stanza}
		di quanto sperammo\verseline
		di quanto credemmo\verseline
		lasciai andare la rotta\verseline
		né mi resta ricordo\verseline
		se non di spechi senz’occhi\verseline
		aperti sull’erebo
	\end{stanza}

	\begin{stanza}
		e tu ancora\verseline
		urgi il ritorno\verseline
		alle spiagge remote\verseline
		agli scogli\verseline
		alla spuma fecondata\verseline
		e irragionevole
	\end{stanza}

	\begin{stanza}
		è lontana da qui Itaca
	\end{stanza}

	\begin{stanza}
		eppure
	\end{stanza}
\end{poem}

\clearpage

\poemtitle{mare i}

\begin{poem}
	\begin{stanza}
		crespo azzurro teso\verseline
		panno steso d’azzurro\verseline
		alle bande d’acciaio
	\end{stanza}

	\begin{stanza}
		supremazia d’un elemento\verseline
		spiegato\verseline
		tra lembi sterili d’un altro
	\end{stanza}

	\begin{stanza}
		ci sia feconda\verseline
		la tua collera
	\end{stanza}

	\begin{stanza}
		ci sia magistero\verseline
		il metro nascosto\verseline
		dei tuoi capricci
	\end{stanza}

	\begin{stanza}
		ci sia fondamento\verseline
		di tutti i giorni\verseline
		attingere in te\verseline
		scaturigine e termine
	\end{stanza}

	\begin{stanza}
		ci sia dato
		per sapienza estrema\verseline
		di riconoscere\verseline
		nel tuo esigere\verseline
		nel tuo suggere\verseline
		nel tuo uccidere\verseline
		il saldo d’un debito
	\end{stanza}

	\begin{stanza}
		ci sia permesso\verseline
		di amare in te\verseline
		la tua crudeltà\verseline
		il tuo nascondere\verseline
		il tuo disvelare\verseline
		il tuo farti\verseline
		itinerario
	\end{stanza}
\end{poem}

\clearpage

\poemtitle{generazione di mostri}

\begin{poem}
	\begin{stanza}
		per primi germinasti\verseline
		i mostri azzurri\verseline
		secreti di grembo estroso\verseline
		che era ignara ancora\verseline
		dei vomeri la terra\verseline
		anomica ancora\verseline
		molto prima che tutta intera\verseline
		si afformicolasse di uomini
	\end{stanza}
\end{poem}

\clearpage

\poemtitle{generazione di dei}

\begin{poem}
	\begin{stanza}
		secondi generasti gli dei
	\end{stanza}

	\begin{stanza}
		parole non invocate\verseline
		parole vergini li facesti\verseline
		quiete incorrotta\verseline
		crudeltà ripiegata\verseline
		senza vittime ancora\verseline
		senza templi\verseline
		né tempo
	\end{stanza}

	\begin{stanza}
		ché ne inscrivesti le vite\verseline
		in troppo vasto circolo\verseline
		vite di cristalli che mutano\verseline
		lentamente
	\end{stanza}

	\begin{stanza}
		che non sanno l’andare
	\end{stanza}
\end{poem}

\clearpage

\poemtitle{generazione di uomini}

\begin{poem}
	\begin{stanza}
		ci ripensasti
	\end{stanza}

	\begin{stanza}
		li avevi fatti ignari\verseline
		di colpa gli dei\verseline
		quasi fuori del nomos
	\end{stanza}

	\begin{stanza}
		e così germinasti\verseline
		gli uomini\verseline
		né spuma né abisso\verseline
		né aureo medio\verseline
		tra mostro e dio\verseline
		ma in bilico\verseline
		non stranieri alle vette\verseline
		al tartaro neppure stranieri
	\end{stanza}

	\begin{stanza}
		ma disposti ad andare\verseline
		i fianchi serrati\verseline
		i sandali ai piedi\verseline
		non ignari di urgenza
	\end{stanza}
\end{poem}

\clearpage

\poemtitle{scoperta della terra}

\begin{poem}
	\begin{stanza}
		solchi di ombra nella terra\verseline
		solchi nella carne gravida\verseline
		sole estrae ombre gialle\verseline
		vapori sospiri di bestia
	\end{stanza}

	\begin{stanza}
		la sera ingiungono partenza\verseline
		i vapori che cercano il cielo\verseline
		nati siamo stranieri alla terra\verseline
		messe di tutte le terre
	\end{stanza}

	\begin{stanza}
		e ancora sapore di morsi\verseline
		buona carne del mio sangue\verseline
		e dei miei morsi nella carne\verseline
		uguale me al sole evocante vapori
	\end{stanza}
\end{poem}

\clearpage

\poemtitle{nomos}

\begin{poem}
	\begin{stanza}
		sovrano Nomos\verseline
		amplesso eterno di tutti gli esseri\verseline
		di ognuno egualmente governi il respiro vitale\verseline
		a nessuno risparmiando la cognizione\verseline
		di una soggezione dappertutto eguale\verseline
		ma più dura ai mortali
	\end{stanza}
\end{poem}

\clearpage

\poemtitle{mare ii}

\begin{poem}
	\begin{stanza}
		fatto marea\verseline
		di lava cobalto\verseline
		monti pian piano\verseline
		come sonno che stemperi\verseline
		il confine della coscienza
	\end{stanza}

	\begin{stanza}
		ti concedi\verseline
		cautamente\verseline
		all’ambiguità\verseline
		o la ritrovi
	\end{stanza}

	\begin{stanza}
		e sei cavità mite\verseline
		e desiderio erto\verseline
		ventre senza confini\verseline
		e impeto di voglia
	\end{stanza}

	\begin{stanza}
		pronto a poco a poco\verseline
		al tuo desiderio\verseline
		ad accoglierti\verseline
		a conoscerti\verseline
		a te stesso
	\end{stanza}
\end{poem}

\clearpage

\poemtitle{cicladi}

\begin{poem}
	\begin{stanza}
		asciugandoti\verseline
		t’irrigidisci\verseline
		contraendoti\verseline
		ti screpoli\verseline
		in mille grani\verseline
		esplodi\verseline
		in schegge\verseline
		di granata\verseline
		rapprese\verseline
		in marmo alieno\verseline
		troppo bianco\verseline
		in aghi duri di sole\verseline
		tra stecche socchiuse
	\end{stanza}

	\begin{stanza}
		ti si levigano\verseline
		le coste ignote\verseline
		in liscia vigilanza\verseline
		in candido logos\verseline
		in autocoscienza\verseline
		frammenti di te\verseline
		che lasci andare\verseline
		subito meno tuoi\verseline
		benché\verseline
		a dirla tutta ti ci\verseline
		nuotano sopra come\verseline
		fave nel brodo
	\end{stanza}
\end{poem}

\clearpage

\poemtitle{argonauti}

\begin{poem}
	\begin{stanza}
		così nuovo il mondo\verseline
		ha già confine
	\end{stanza}

	\begin{stanza}
		o limite o termine\verseline
		estremo bordo\verseline
		recinto\verseline
		frontiera
	\end{stanza}

	\begin{stanza}
		misura
	\end{stanza}

	\begin{stanza}
		oltre cui non va\verseline
		neppure la lancia scagliata\verseline
		da un passo\verseline
		ma dicono resti confitta nell’aria
	\end{stanza}

	\begin{stanza}
		vado
	\end{stanza}

	\begin{stanza}
		lascio\verseline
		il mio campo\verseline
		raccolgo brigata scelta di compagni\verseline
		preziosa agli dei gioventù\verseline
		se possa servire a obbligarne i favori\verseline
		un dono di carne lungo la strada
	\end{stanza}

	\begin{stanza}
		poi tornare indietro a contare\verseline
		magari con meno abbondandante\verseline
		messe d’eroi\verseline
		magari lasciata cadere\verseline
		tributo all’andare\verseline
		mia pelle già stata\verseline
		già storia\verseline
		io stesso mutato
	\end{stanza}

	\begin{stanza}
		è mio il viaggio\verseline
		mio l’andare\verseline
		e cosa esigerà il mare\verseline
		in cambio di lasciarmi passare\verseline
		che io non possa dargli?
	\end{stanza}

	\begin{stanza}
		io voglio\verseline
		strofinare il ventre\verseline
		delle mie navi sul suo\verseline
		ventre antico
	\end{stanza}

	\begin{stanza}
		e poi non è vero che il mondo\verseline
		sia nuovo\verseline
		così tanto
	\end{stanza}
\end{poem}

\poemtitle{stasimo i}

\begin{poem}
	\begin{stanza}
		due fili gemelli ci sono\verseline
		e uno è presso il confine\verseline
		i due s’intrecciano\verseline
		senza stridere
	\end{stanza}

	\begin{stanza}
		due rivoli d’acqua ci sono\verseline
		gemelli che cercano il mare\verseline
		e uno è presso il confine\verseline
		i due si mescolano\verseline
		nascostamente\verseline
		sotterranee acque
	\end{stanza}

	\begin{stanza}
		e ogni foglia giovinetta sui rami\verseline
		del mio campo\verseline
		ha l’eguale presso il confine\verseline
		e così ogni petalo e spine\verseline
		di rose selvatiche
	\end{stanza}

	\begin{stanza}
		di ogni specie vivente\verseline
		pesci e uccelli e serpenti tra l’erba\verseline
		di ogni specie due gemelli vivono\verseline
		uno qui accanto a me\verseline
		e l’altro presso il confine\verseline
		accordando entrambi la rapida vita\verseline
		su un modo comune
	\end{stanza}

	\begin{stanza}
		e di profumi e di balsami dolci\verseline
		di spezie e d’unguenti d’Oriente\verseline
		che si danno l’oncia al prezzo\verseline
		di una notte\verseline
		e d’armonie di cimbali e d’àuli\verseline
		e di versi misurati dall’aedo\verseline
		e di veli di regine\verseline
		e di maschere di morti\verseline
		nulla esiste che non abbia
		presso il confine il suo gemello
	\end{stanza}

	\begin{stanza}
		luce di candele\verseline
		e le ombre\verseline
		guizzanti sui muri\verseline
		e calici di vino\verseline
		accanto a una caccia copiosa\verseline
		accanto al trofeo di un carniere rigonfio
	\end{stanza}

	\begin{stanza}
		e conchiglie e gabbiani e stelle marine\verseline
		gemelli da esibire in reti gemelle\verseline
		e alberi di navigli e sartie e fasciame\verseline
		perduti sul mare...
	\end{stanza}

	\begin{stanza}
		io stesso fui\verseline
		conchiglia delicata\verseline
		muto pesce nel mare\verseline
		pianta e uccello\verseline
		e fanciulla e ragazzo\verseline
		ma ora sento la mia vita\verseline
		ridiscendere la china\verseline
		inseguire le origini
	\end{stanza}
\end{poem}

\clearpage

\poemtitle{ellade}

\begin{poem}
	\begin{stanza}
		cocci nell’arco luminoso del mattino\verseline
		caleidoscopio di sprazzi rifratti\verseline
		dispendio di seme splendente nella luce
	\end{stanza}

	\begin{stanza}
		poi frammenti di nero e di rosso\verseline
		dispersi nel mare
	\end{stanza}

	\begin{stanza}
		la terra\verseline
		quella pesante\verseline
		quella ferma\verseline
		ma strattonata dal mare\verseline
		più verso il basso\verseline
		la terra\verseline
		ha ogni pietra dipinta di sangue\verseline
		di roba\verseline
		splancnica\verseline
		di bile nera
	\end{stanza}

	\begin{stanza}
		ancora difficile o retrospettivamente\verseline
		millantato\verseline
		il sorgere del logos
	\end{stanza}
\end{poem}

\clearpage

\poemtitle{minotauro}

\begin{poem}
	\begin{stanza}
		ogni andito\verseline
		ogni piega\verseline
		ogni diverticolo\verseline
		del labirinto\verseline
		promettono fuga\verseline
		ma insidiosi mi tengono\verseline
		tra le mura della tana\verseline
		levigate\verseline
		immutabili\verseline
		conosciute
	\end{stanza}

	\begin{stanza}
		ormai non conto più i giorni\verseline
		finché si incastrino in schema\verseline
		sempre eguale le costellazioni\verseline
		finché colmino il perimetro\verseline
		del mio groviglio\verseline
		gli stridi di fanciulli impuberi\verseline
		due volte sette\verseline
		femmine e maschi in parti eguali\verseline
		e quasi\verseline
		ignari di sesso
	\end{stanza}

	\begin{stanza}
		si aspettano che ne faccia scempio\verseline
		e mi fuggono addosso\verseline
		inondati da paura di bestie
	\end{stanza}

	\begin{stanza}
		sempre eguale terrore\verseline
		se ne colmano gli sguardi\verseline
		sempre eguale\verseline
		gli schiaffeggia le nari\verseline
		la fragranza della mia carne\verseline
		sempre eguale a essere\verseline
		troppo pronta a dargli retta
	\end{stanza}

	\begin{stanza}
		loro già edotti che il labirinto\verseline
		sa troppi ingressi e nessun esito\verseline
		che non sia la mia tana\verseline
		che non sia la mia carne\verseline
		né di costoro è concesso liberarmi
	\end{stanza}

	\begin{stanza}
		si aspettano che ne faccia scempio\verseline
		e mi fuggono addosso\verseline
		inondati da voglia di bestie\verseline
		e un lampo di sfida o trionfo\verseline
		negli sguardi e lo indovinano\verseline
		che magari gli invidio\verseline
		la rapida fine
	\end{stanza}

	\begin{stanza}
		la loro carne che già si apre\verseline
		pressata dalla mia sempre\verseline
		troppo prevedibilmente ridesta\verseline
		da terrore o voglia di bestia\verseline
		e intanto ho perso il conto\verseline
		delle volte e degli squarci
	\end{stanza}

	\begin{stanza}
		attendo soltanto\verseline
		chi il nomos comandi di spezzare\verseline
		la ciclicità di questo tempo\verseline
		perso nel groviglio di me\verseline
		attendo soltanto ormai\verseline
		un punto e basta
	\end{stanza}

	\begin{stanza}
		e poi il terrore a mia volta\verseline
		mi prende e mi chiedo\verseline
		se non sia già giunto\verseline
		se non sia perso anche lui\verseline
		nel suo labirinto\verseline
		se non sia più questo il tempo\verseline
		o il luogo giusto per attenderlo\verseline
		se non siano anche i suoi occhi\verseline
		dilatati da sensi bestiali\verseline
		o se non venga qui per irridermi\verseline
		giorno per giorno
	\end{stanza}

	\begin{stanza}
		lascio a marcire\verseline
		carogne di gioventù\verseline
		chiedendo che lo trascini\verseline
		fin quaggiù la voglia\verseline
		se ne abbia di eguali alle mie
	\end{stanza}

	\begin{stanza}
		e già si approssima la sizigia\verseline
		degli astri nel cielo\verseline
		e già ho deciso\verseline
		rifiuterò stavolta il pasto\verseline
		rifiuterò lo scempio\verseline
		che non sia forse\verseline
		tra i due volte sette giovani\verseline
		il liberatore
	\end{stanza}

	\begin{stanza}
		ma se così non sia\verseline
		risparmiandoli tutti\verseline
		li farò miei compagni nell’attesa\verseline
		riservando loro una sorte\verseline
		certo peggiore\verseline
		eguale alla mia
	\end{stanza}
\end{poem}

\clearpage

\poemtitle{mare iii}

\begin{poem}
	\begin{stanza}
		rimpiattino di piaghe\verseline
		sparse di sabbia\verseline
		di sale di schegge\verseline
		di carapaci spezzati
	\end{stanza}

	\begin{stanza}
		intrecci di luci e cupezze\verseline
		tela franta dei fondali\verseline
		rubrica di coralli\verseline
		toccamenti di cnidarie
	\end{stanza}

	\begin{stanza}
		farandole di alghe\verseline
		astuzie di polpi\verseline
		matte cavalcate\verseline
		di cavalli di mare
	\end{stanza}

	\begin{stanza}
		quali mostri increati\verseline
		insidie mirabili\verseline
		pensi ancora nei gorghi\verseline
		quanti ne ascondi?
	\end{stanza}

	\begin{stanza}
		ma però quanta\verseline
		infinita voglia di esplorarti\verseline
		ci hai scritto tu dentro\verseline
		che ti cercassimo\verseline
		fino alle coste remote\verseline
		fino ai fondali del profondo?
	\end{stanza}
\end{poem}

\clearpage

\poemtitle{stasimo ii}

\begin{poem}
	\begin{stanza}
		nessun gesto è più antico\verseline
		che impastare il pane\verseline
		meno che mettere a dormire\verseline
		le ossa dei morti\verseline
		meno che dar sollievo alla carne\verseline
		se chieda raschiamenti ulteriori
	\end{stanza}

	\begin{stanza}
		nessun gesto è più antico\verseline
		che affondare il coltello\verseline
		nella carne calda d’un uomo\verseline
		che sia nemico o compagno\verseline
		o giovane vergine dio\verseline
		da restituire ai suoi pari
	\end{stanza}

	\begin{stanza}
		si mangia poi\verseline
		per assorbire\verseline
		per comprendere\verseline
		perché non vada perduto midollo vitale\verseline
		né si sciolga continuità tra i mortali
	\end{stanza}

	\begin{stanza}
		sono lo stesso pietà e violenza
	\end{stanza}
\end{poem}

\clearpage

\poemtitle{limite}

\begin{poem}
	\begin{stanza}
		terre\verseline
		protese nel mare\verseline
		manciate di sassi\verseline
		ambiguità di frontiere\verseline
		ogni momento riplasmate\verseline
		dal gioco dei flutti
	\end{stanza}

	\begin{stanza}
		come non lasciarsi\verseline
		crescere e nutrire\verseline
		dall’idea di avere di fronte\verseline
		dall’idea di tendere\verseline
		verso l’altro e l’altrove?
	\end{stanza}

	\begin{stanza}
		coscienza nascente\verseline
		di un’umanità\verseline
		che da ogni parte affolla le rive\verseline
		e incrocia cammini\verseline
		e tesse incontri\verseline
		sul mare
	\end{stanza}
\end{poem}

\clearpage

\poemtitle{mare iv}

\begin{poem}
	\begin{stanza}
		maestà di tutte le cose\verseline
		che sono\verseline
		tu tutto del cosmo\verseline
		in te riposo\verseline
		signore luminoso
	\end{stanza}

	\begin{stanza}
		oltre le pluralità apparenti\verseline
		tu insegni che uno è ciò che è
	\end{stanza}

	\begin{stanza}
		sovrano di tutte le cose\verseline
		che scorrono\verseline
		origine di ogni movimento\verseline
		fulcro di ogni ciclo\verseline
		tu tutto produci\verseline
		a tutto dai incremento\verseline
		tutto accogli nel tuo grembo\verseline
		nel termine che il nomos decreta
	\end{stanza}

	\begin{stanza}
		su di te s’impernia la distesa\verseline
		della terra\verseline
		e l’uno è all’altra\verseline
		vicendevolmente\verseline
		complemento e sponda\verseline
		necessità e sostegno\verseline
		e scelta
	\end{stanza}
\end{poem}

\clearpage

\poemtitle{sirene}

\begin{poem}
	\begin{stanza}
		seduzione\verseline
		sviamento\verseline
		naufragio
	\end{stanza}

	\begin{stanza}
		volle trovare in noi sguardo d’uomo
	\end{stanza}

	\begin{stanza}
		ma noi eravamo semplicemente\verseline
		attesa
	\end{stanza}

	\begin{stanza}
		ora mutarsi in schiuma\verseline
		non altra via che questa\verseline
		ché tanto\verseline
		tutto muore nel mare\verseline
		e rivive
	\end{stanza}
\end{poem}

\clearpage

\poemtitle{odisseo}

\begin{poem}
	\begin{stanza}
		le lasciai da un pezzo le Sirene\verseline
		sono ancora là temo\verseline
		prigioniere di un prato fiorito\verseline
		verrà anche per loro l’incontro\verseline
		e la consapevolezza?
	\end{stanza}

	\begin{stanza}
		quanto a me\verseline
		non so se confessare che\verseline
		ho nostalgia di quei giorni
	\end{stanza}

	\begin{stanza}
		ormai è parecchio che vado per isole\verseline
		che mi perdo nel cabotaggio piccolo
	\end{stanza}

	\begin{stanza}
		l’approdo lo tengo lontano\verseline
		ci son nato sì\verseline
		ci ho conosciuto donna\verseline
		ho generato
	\end{stanza}

	\begin{stanza}
		ma è troppo grande il mondo là fuori\verseline
		troppa febbre brucia ancora di vita\verseline
		e Itaca è l’approdo e la pace
	\end{stanza}

	\begin{stanza}
		ho imparato ad apprezzare quel che si dà\verseline
		di umano tra due che s’incontrano per caso\verseline
		in un mercato di terre lontane
	\end{stanza}

	\begin{stanza}
		lo sguardo d’intesa\verseline
		una stretta di mano virile\verseline
		il calore dell’accoglienza\verseline
		e anche la furbizia e il raggiro\verseline
		tutto quanto c’è di umano negli uomini
	\end{stanza}

	\begin{stanza}
		come esser soli\verseline
		se ogni uomo\verseline
		ogni straniero\verseline
		lo saluti compagno\verseline
		alla smania che ti agita il sangue\verseline
		se senti tua ogni terra\verseline
		affacciata su un mare comune?
	\end{stanza}
\end{poem}

\clearpage

\poemtitle{calipso}

\begin{poem}
	\begin{stanza}
                agli antichi tempi\verseline
                appartiene il tuo amore\verseline
                quando altri dèi altro sole\verseline
                presiedevano al destino
	\end{stanza}

	\begin{stanza}
                però qui il sonno\verseline
                di volta a volta\verseline
                farmaco prezioso\verseline
                ladro di sperienze\verseline
                scambia con il dolore oblio\verseline
                con vita altra vita
	\end{stanza}

	\begin{stanza}
                vorrei che morisse il tuo amore\verseline
                e poi nel giorno nuovo\verseline
                rinascesse intatto\verseline
                come un granaio sigillato
	\end{stanza}

	\begin{stanza}
                però qui il tempo\verseline
                s’avvolge qualche volta\verseline
                in riccioli o spire\verseline
                e sottrae agli sguardi\verseline
                il suo gioco segreto
	\end{stanza}

	\begin{stanza}
                elisir di sole di terra d’erbe\verseline
                mia patria trasmutata\verseline
                che a me ti nascondevi\verseline
                a dodici passi appena\verseline
                t’ho ritrovata\verseline
                son vicino a perderti ancora
	\end{stanza}

	\begin{stanza}
                però io disperso\verseline
                in un letto a difendermi da tedio\verseline
                o peggio da carezze\verseline
                solventi sleali\verseline
                segno della tua cura\verseline
                o a ritrovarmi in autocratici gesti
	\end{stanza}

	\begin{stanza}
                mille volte l’ho passata\verseline
                stanotte la frontiera\verseline
                misticanza di sonno e veglia\verseline
                esplosione di frammenti
	\end{stanza}

	\begin{stanza}
                quali ragioni dentro questa smania\verseline
                di cercare sempre nel presente\verseline
                un andersh una fuga\verseline
                un essere diverso\verseline
                il permesso di vivere portandomi\verseline
                un segno segreto sottopelle?
	\end{stanza}
\end{poem}

\clearpage

\poemtitle{eroe}

\begin{poem}
	\begin{stanza}
		immergere mani\verseline
		in polle grosse\verseline
		delle profondità\verseline
		in quelle gravi\verseline
		di sanie
	\end{stanza}

	\begin{stanza}
		urgere ansiti\verseline
		da escrescenze\verseline
		livide\verseline
		astiose\verseline
		di polpo stanato\verseline
		estrarne stille
	\end{stanza}

	\begin{stanza}
		sempre prima snudarsi\verseline
		ma ditemi dove resta l’eroe\verseline
		lasciato a mezzarsi\verseline
		a estri altrui?
	\end{stanza}

	\begin{stanza}
		o magari insegnare a queste serpi\verseline
		di dita maestria di studi\verseline
		orgoglio di esecuzione\verseline
		autogestita
	\end{stanza}
\end{poem}

\clearpage

\poemtitle{stasimo iii}

\begin{poem}
	\begin{stanza}
                dei giorni ci siano preziose\verseline
                anche queste cose\verseline
                lo sguardo duro dell’avversario\verseline
                la gemma amara dell’ingiustizia\verseline
                la parola che scopre un compagno nell’uomo\verseline
                che ci passa accanto
	\end{stanza}
\end{poem}

\clearpage

\poemtitle{padre o lo straniero}

\begin{poem}
	\begin{stanza}
		da non lasciarsi conoscere\verseline
		se non per via di violenza\verseline
		se non da chi ignora\verseline
		che sia vergogna\verseline
		che sia timore\verseline
		che sia angoscia\verseline
		che sia censura\verseline
		la sua carne
	\end{stanza}

	\begin{stanza}
		troppo coriacea\verseline
		troppo antica\verseline
		troppo sedimentata\verseline
		la geologia di quella virilità\verseline
		perché ne sia questione lieve
	\end{stanza}

	\begin{stanza}
		ma che sia\verseline
		carne di uomo la sua carne\verseline
		e non granito senza speranza\verseline
		lo seppi già varcata la frontiera\verseline
		con Itaca alle spalle\verseline
		all’arco delle sue spalle\verseline
		ritrose quando non guardi\verseline
		e tenere e lo seppi\verseline
		alle sue mani e lo seppi ancora\verseline
		alla sua carne da dischiudere\verseline
		per uscirne alla luce
	\end{stanza}
\end{poem}

\clearpage

\poemtitle{terra}

\begin{poem}
	\begin{stanza}
		carne di tutti costoro\verseline
		che lontani da Itaca\verseline
		o risparmiati\verseline
		o rectius sputati dal mare\verseline
		si abbracciano alle rene\verseline
		di questa terra
	\end{stanza}

	\begin{stanza}
		carne salda e vita sapiente\verseline
		schermaglie virili di mani di sguardi\verseline
		scambiati per strada\verseline
		parole di succo appena offuscate\verseline
		dal crescere lentissimo dei giorni\verseline
		pienezza di non essere io uno soltanto\verseline
		ma parte di molti\verseline
		molti io stesso
	\end{stanza}
\end{poem}

\clearpage

\poemtitle{mare v}

\begin{poem}
	\begin{stanza}
		dall’alto vorrei saperti\verseline
		come albatro o gabbiano\verseline
		che abita le tempeste\verseline
		che non teme canto di risacca
	\end{stanza}

	\begin{stanza}
		dall’alto vorrei tracciarti\verseline
		i confini infinitamente intarsiati\verseline
		le tue curve frattali\verseline
		più dolci delle ànche dell’amato
	\end{stanza}

	\begin{stanza}
		sono grani d’uva Tiro\verseline
		ed Efeso e Smirne\verseline
		e Cuma e Alessandria\verseline
		grani d’uva di Corinto\verseline
		afformicolati di uomini
	\end{stanza}

	\begin{stanza}
		sono cascate di guizzi\verseline
		sul crespo dell’acqua\verseline
		fremiti di corde di cetra\verseline
		e campi di croquet\verseline
		di salti di delfini
	\end{stanza}

	\begin{stanza}
		sono amabili\verseline
		anche i mostri del tuo es\verseline
		anche le collere\verseline
		che ti corrono nel ventre
	\end{stanza}

	\begin{stanza}
		e dall’alto vorrei vederti\verseline
		e parlare di te ai compagni\verseline
		dall’alto magari d’un paio\verseline
		d’ali rubate ai gabbiani\verseline
		o artatamente divisate\verseline
		da un padre\verseline
		mastro d’inciarmi
	\end{stanza}
\end{poem}

\clearpage

\poemtitle{stasimo iv}

\begin{poem}
	\begin{stanza}
		non è difficile sopravvivere\verseline
		del passato è sufficiente\verseline
		scordare fallimenti delusioni\verseline
		ed errori e sofferenze\verseline
		occasioni perdute e gli amori\verseline
		sciupati e i piaceri non saputi
	\end{stanza}

	\begin{stanza}
		non è difficile sopravvivere\verseline
		del domani non ammettere paure\verseline
		esibirsi indifferenti se all’orizzonte\verseline
		s’approssima il fantasma della vecchiaia\verseline
		come sera copre lungamente\verseline
		i passi dei monti
	\end{stanza}

	\begin{stanza}
		non è difficile sopravvivere\verseline
		se un altro giorno si sa\verseline
		s’aggiunge agli altri passati\verseline
                e dunque ignoriamo la fame\verseline
                e il sonno e la fatica\verseline
                straziamoci le carni\verseline
                affinché imparino a non concedersi\verseline
                al desiderio
	\end{stanza}

	\begin{stanza}
                non ci è permesso ricordare\verseline
                che è umana stirpe la nostra\verseline
                debole vulnerabile\verseline
                ma suscitiamo a noi stessi\verseline
                una corazza d'acciaio\verseline
                impenetrabile più che scudo d'Achille
	\end{stanza}
\end{poem}

\clearpage

\poemtitle{approdi?}

\begin{poem}
	\begin{stanza}
		eppure mare c’era scritto\verseline
		nella carne dell’uomo\verseline
		non polvere\verseline
		né terra impastata a sudore\verseline
		ma acqua\verseline
		acqua salsa\verseline
		acqua di mare
	\end{stanza}

	\begin{stanza}
		e allora non ci è permesso ricordare\verseline
		che è umana stirpe la nostra\verseline
		umane vulnerabile stirpe
	\end{stanza}

	\begin{stanza}
		e ignoriamo la fame\verseline
		e sonno e fatica\verseline
		straziamoci le carni\verseline
		affinché imparino a non concedersi\verseline
		alla voglia di stare
	\end{stanza}

	\begin{stanza}
		nulla importa il gravame dei giorni\verseline
		pur di sfiorare le rive e andare\verseline
		non prima però di aver visto\verseline
		empori fenici e barattate\verseline
		ambre e conchiglie e saputi\verseline
		di prima mano i modi\verseline
		di amarsi nei porti sparsi\verseline
		per le rive ma poi andare
	\end{stanza}

	\begin{stanza}
		non per tornarne cospicui\verseline
		da pletorici bovi ammansiti\verseline
		come prescrivono i saggi\verseline
		a Itaca di pecunia o esperienza\verseline
		o dottrina o perché ne sia\verseline
		meglio pasciuta l’età grave\verseline
		sazia nel letto stato in caldo
	\end{stanza}

	\begin{stanza}
		ma perché tracciare solchi\verseline
		con le rapide navi sull’acqua\verseline
		perché andare oltre il giro\verseline
		stretto della terra\verseline
		oltre la solidità\verseline
		cara sotto i piedi\verseline
		è destino d’uomo
	\end{stanza}

	\begin{stanza}
		né conta che sia di piccolo\verseline
		cabotaggio la rotta di Odisseo\verseline
		o di mare alto
	\end{stanza}

	\begin{stanza}
		perché andare ancora e ancora\verseline
		e sì dispiegare ovunque astuzie\verseline
		e voglia e tracotanza e male\verseline
		a star lontani è umano\verseline
		ma prestare orecchio\verseline
		al canto di un altrove\verseline
		onnipresente\verseline
		è destino d’uomo
	\end{stanza}

	\begin{stanza}
		perché andare per andare\verseline
		è destino di quanti ci sono mortali\verseline
		e lasciarne i più fortunati\verseline
		al mare gravame di carne\verseline
		quando nomos fa essere sera
	\end{stanza}

	\begin{stanza}
		sacertà di quanto eternamente muta\verseline
		eternamente a sé eguale\verseline
		che non sanno misurare né il tempo\verseline
		stabilito dalla madre\verseline
		né l’esasperante incespicare dei secoli\verseline
		né l’avvicendarsi piatto di stagioni\verseline
		di opere e di giorni
	\end{stanza}

	\begin{stanza}
		era scritto mare in noi\verseline
		e doveva essere\verseline
		mare
	\end{stanza}
\end{poem}
