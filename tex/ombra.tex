\begin{volumetitlepage}
	\volumetitle{Ombra}
	\volumeheader{Ombra}
	\bigskip\bigskip\bigskip
	\volumeepigraph{
		\begin{otherlanguage}{romanian}
			Nu mă regăsesc în el, deşi îmi pare evidentă prezenţa isteriei mele de atunci.
		\end{otherlanguage}
	}
	\volumeattribution{Emil Cioran}
\end{volumetitlepage}

\poemtitle{i}

\begin{artItem}
	Affreschi della villa di Livia a Prima Porta
\end{artItem}

\begin{poem}
	\begin{stanza}
		andai per starci (poco)\verseline
		ispido nelle prime mattine di gennaio\verseline
		ma quanto tempo per farne\verseline
		una dolce dorata coppa di solitudine\verseline
		una piccola custodia di rifugi cesellati a mano?\verseline
		e poi ebbi ombra\verseline
		di una primavera troppo lunga, forse\verseline
		presi una di quelle primaverili sere di agosto\verseline
		e me ne andai
	\end{stanza}
\end{poem}

\clearpage

\poemtitle{ii}

\begin{artItem}
	Umberto Boccioni, Quelli che restano, seconda versione
\end{artItem}

\begin{poem}
	\begin{stanza}
		sul bordo d'estate mi torce i sensi\verseline
		più amara più tesa nell'assenza\verseline
		l'aria del paese dove nacqui adulto
	\end{stanza}

	\begin{stanza}
		ma fui lesto a privarmene\verseline
		consentaneo al gioco di incontri\verseline
		e abbandoni che io chiamo vita
	\end{stanza}
\end{poem}

\clearpage

\poemtitle{iii}

\begin{poem}
	\begin{stanza}
		terra hai segreti\verseline
		sei dolce e petrosa\verseline
		e poi le cavità di mare\verseline
		onde non sei sprovvista\verseline
		quando ti vennero?
	\end{stanza}

	\begin{stanza}
		così ti prendi il senno\verseline
		dei figli e uguale di chi\verseline
		t’ha saputa per avventura
	\end{stanza}
\end{poem}

\clearpage

\poemtitle{iv}

\begin{poem}
	\begin{stanza}
		golfi lunghi\verseline
		orlati di rapido asfalto\verseline
		sorpresi ancora dall’acqua\verseline
		come ventri ricolmi\verseline
		erano forse contrade e casali\verseline
		ora sono nascondimento segreto\verseline
		della mia terra
	\end{stanza}
\end{poem}

\clearpage

\poemtitle{v}

\begin{poem}
	\begin{stanza}
		fanciullezza ombrosa\verseline
		fanciullezza soltanto immaginata\verseline
		costruita nel ricordo\verseline
		non irraggiungibile certo\verseline
		ma lasciata da parte\verseline
		ferocemente
	\end{stanza}

	\begin{stanza}
		cavità dolci e strette della mia terra\verseline
		dolci profonde cavità della mia terra\verseline
		custodite per me vi prego\verseline
		il segreto del vostro altrove
	\end{stanza}

	\begin{stanza}
		dischiudetelo per me soltanto\verseline
		e lasciatemi l’orgoglio\verseline
		della mia appartenenza\verseline
		che possa portarmelo\verseline
		in giro per il mondo
	\end{stanza}

	\begin{stanza}
		io sono vostro iure sanguinis\verseline
		e guai a chi mi dice mentitore
	\end{stanza}
\end{poem}

\clearpage

\poemtitle{vi}

\begin{artItem}
	Adolph Menzel, \begin{otherlanguage}{german}%
		Das Balkonzimmer%
	\end{otherlanguage}
\end{artItem}

\begin{poem}
	\begin{stanza}
		e nidi caldi d’ombra\verseline
		tra le case vecchie\verseline
		e negli angoli delle nuove\verseline
		che anche loro hanno segreti
	\end{stanza}

	\begin{stanza}
		ci si ferma a respirare\verseline
		odore d’incenso\verseline
		profumo di letto
	\end{stanza}
\end{poem}

\clearpage

\poemtitle{vii}

\begin{artItem}
	Mario Sironi, Paesaggio urbano 1921
\end{artItem}

\begin{poem}
	\begin{stanza}
		alla stazione quella prima volta\verseline
		stavo come chi aspetta di nascere
	\end{stanza}

	\begin{stanza}
		poi il cenno d’una strada ignota\verseline
		un mistero che sorveglia le case
	\end{stanza}

	\begin{stanza}
		era sera e li ricordo i fuochi vivi\verseline
		le foglie stropicciate e allegre
	\end{stanza}

	\begin{stanza}
		ed erano mio padre e mia madre\verseline
		quella città appena incontrata
	\end{stanza}

	\begin{stanza}
		licenza di vivere nuovo\verseline
		promessa di interstizi tra il cemento
	\end{stanza}
\end{poem}

\clearpage

\poemtitle{viii}

\begin{artItem}
	Felice Casorati, Case popolari
\end{artItem}

\begin{poem}
	\begin{stanza}
		uno scaffale nell'angolo\verseline
		stretto d'una gelateria\verseline
		la scatola un po' polverosa\verseline
		un po' piena di meraviglie\verseline
		e lucente di armi smaltate
	\end{stanza}

	\begin{stanza}
		i piccoli lampioni appena accesi\verseline
		nella foschia luccicante d'autunno
	\end{stanza}

	\begin{stanza}
		i passanti\verseline
		cappelli e cappotti\verseline
		e occhiate ammiccanti
	\end{stanza}

	\begin{stanza}
		e quelle strane e lente sbuffate di vento
	\end{stanza}

	\begin{stanza}
		tra i vocaboli ornati da guglie in pietra sponga\verseline
		tra le vie cittadine strette da cemento dei '60\verseline
		tra le strade che tentano grembi insanguinati tra le colline
	\end{stanza}

	\begin{stanza}
		ma vengo a vedere\verseline
		la vita\verseline
		le vite\verseline
		in centro\verseline
		quando posso
	\end{stanza}
\end{poem}

\clearpage

\poemtitle{ix}

\begin{artItem}
	Gustave Caillebotte, \begin{otherlanguage}{french}%
		Les raboteurs de parquet%
	\end{otherlanguage}
\end{artItem}

\begin{poem}
	\begin{stanza}
		una spolverata di lampioni umidi\verseline
		viali d'oro croccante\verseline
		(ma in questi giorni soltanto)\verseline
		e pomeriggi di miele ambrato\verseline
		tra i condomini
	\end{stanza}
\end{poem}

\clearpage

\poemtitle{x}

\begin{artItem}
	Romano Rui, Elemento verticale
\end{artItem}

\begin{poem}
	\begin{stanza}
		vengo a concedermi\verseline
		—perché no?—\verseline
		un'andata nell'aria asprigna\verseline
		e dolce e voltare\verseline
		per gioco le pagine\verseline
		da accartocciare vermiglie\verseline
		tra i passi
	\end{stanza}

	\begin{stanza}
		a cercare ancora un poco di miele\verseline
		sull'orlo della sera\verseline
		a godere ai mille\verseline
		e mille spilli d'acciaio\verseline
		sulla mani e il collo\verseline
		a raccogliere il muso al vento\verseline
		ogni pezzo d'autunno
	\end{stanza}
\end{poem}

\clearpage

\poemtitle{xi}

\begin{artItem}
	Enrico Castellani, Spartito
\end{artItem}

\begin{poem}
	\begin{stanza}
		il luogo lo scelsi perché disgiunto\verseline
		nebuloso altrove\verseline
		ma non sapevo\verseline
		avesse un oltre
	\end{stanza}

	\begin{stanza}
		ero andato per starci poco\verseline
		ispido nelle mattine di gennaio
	\end{stanza}

	\begin{stanza}
		poi mi bastò una strada\verseline
		nell'ombra di giugno\verseline
		il gioco segreto del sole
	\end{stanza}

	\begin{stanza}
		la sera soltanto riposare distesi\verseline
		sulla pietra calda alla stazione
	\end{stanza}

	\begin{stanza}
		solo dopo ebbi ombra\verseline
		di una troppo lunga primavera
	\end{stanza}
\end{poem}

\clearpage

\poemtitle{xii}

\begin{artItem}
	Honoré Daumier, \begin{otherlanguage}{spanish}%
		Don Quixote%
	\end{otherlanguage}
\end{artItem}

\begin{poem}
	\begin{stanza}
		fu avara di brama\verseline
		e rada e arresa\verseline
		la vita lassù\verseline
		finché restai
	\end{stanza}

	\begin{stanza}
		ma fu mia sorgente\verseline
		non quella ingombrante\verseline
		carnale
	\end{stanza}

	\begin{stanza}
		allora ancora non sapevo\verseline
		partenza
	\end{stanza}
\end{poem}
