\poemtitle{i}

\vspace*{2cm}

	\begin{verse}
		sappiamo dirci per che ragione\\
		ci perdiamo nel negativo di danze\\
		attacchi parate sagacie maligne\\
		spiegamento di difese brutali?
	\end{verse}

	\begin{verse}
		sappiamo dirci per che ragione\\
		ci incastriamo in conversari infiniti\\
		parole-dardo spinte sotto pelle\\
		cretese intrico di assalti e ferite?
	\end{verse}

	\begin{verse}
		sappiamo dirci per che ragione\\
		non c'è più che le teste dell'idra?\\
		lei ancora si pasce dei resti\\
		della nostra morta stagione
	\end{verse}

\clearpage

\poemtitle{ii}

\vspace*{2cm}

	\begin{verse}
		che fregola mi scappa di mattina nel sesso\\
		di trovarmi voglia di meccaniche lusinghe\\
		d'inventarmi smania di dita\\
		imperiose serpi e maestre\\
		des études d'exécution transcendante\\
		che comandano miele alle membra
	\end{verse}

	\begin{verse}
		ma poi che fatica scrivere di cose\\
		indicibilissime\\
		di fame e di sete e di voglia\\
		di questo meccanico amore\\
		autoironico\\
		oh sì è anch'esso amore\\
		ne ha almeno la viltà\\
		e tanticchia di strazio
	\end{verse}

\clearpage

\poemtitle{iii}

\vspace*{2cm}

	\begin{verse}
		incerto il tuo gesto\\
		tra tenerezza e disprezzo\\
		ma forse\\
		non sapendo scegliere\\
		vuol tenere di una cosa\\
		e dell'altra
	\end{verse}

\clearpage

\poemtitle{iv}

\vspace*{2cm}

	\begin{verse}
		come di capriolo già svegliato\\
		(che al mattino sa scrollarsi di dosso\\
		quanto di notturna rugiada gli si sia impigliata)\\
		sfregarsi in fretta il desiderio\\
		apprendendolo alla scorza di un pioppo
	\end{verse}

	\begin{verse}
		così non mi si appanna di struggimenti\\
		sùbiti lo sguardo per strada\\
		né mi sgretola un culo che passa la lealtà\\
		di buona bestia e chiusa
	\end{verse}

	\begin{verse}
		il capriolo che non vuol gravarsi di voglia\\
		quanto è lungo il giorno\\
		s'abbevera in fretta alla polla torbida\\
		pure se non lo trae la sete\\
		che non gliene salga più tardi l'assillo
	\end{verse}

	\begin{verse}
		e comunque anche nell’acqua sporca\\
		non resta sempre preso un po' di sole?
	\end{verse}

\clearpage

\poemtitle{v}

\vspace*{2cm}

	\begin{verse}
		ma amo la venustà sottile che si cela\\
		nell'arco delle tue caviglie\\
		la tua pesantezza da odalisca\\
		che sale le scale
	\end{verse}

	\begin{verse}
		amo i seni che inutilmente copri\\
		nelle chiese (si volta anche il prete)\\
		ma che a me mostri indifferente\\
		al mio digiuno
	\end{verse}

	\begin{verse}
		dovrei allora amare ancora\\
		il tuo sesso? dimmi prima se puoi\\
		del mio sonno d'efebo verecondo\\
		cosa ne hai fatto?
	\end{verse}

\clearpage

\poemtitle{vi}

\vspace*{2cm}

	\begin{verse}
		spesso in questi giorni mi fa compagnia\\
		un oscillare da metronomo\\
		che millanta autenticità\\
		ma poi l'inflessibilità del moto\\
		conduce me al suo esito scontato\\
		e dimentica me lungo la via\\
		un grumo di umana angoscia ad attendermi\\
		però un'abluzione (quando accade)\\
		e un attenuante con me\\
		e si tira oltre
	\end{verse}

\clearpage

\poemtitle{vii}

\vspace*{2cm}

	\begin{verse}
		non portavo armatura\\
		non sai?
	\end{verse}

	\begin{verse}
		neppure un gregge di pecore\\
		avrei saputo guidare alla fonte\\
		o difendere dagli assalti dei lupi\\
		inetto di certo a montare\\
		il più bolso ronzino
	\end{verse}

	\begin{verse}
		perché m'investisti cavaliere\\
		contraria al destino?\\
		perché m'invitasti alla cura\\
		con gesto regale?
	\end{verse}

	\begin{verse}
		io non sapevo
	\end{verse}

\clearpage

\poemtitle{viii}

\vspace*{2cm}

	\begin{verse}
		poi mi va di trattenere dentro\\
		il desiderio come latte\\
		nel cavo delle mammelle\\
		o cadenza non risolta da tanto
	\end{verse}

	\begin{verse}
		e non mi va che mi coli via\\
		per le vie facili\\
		sicché comando alle mani\\
		di conoscere il tufo
	\end{verse}

	\begin{verse}
		e il mare riarso e la lava\\
		e il cinabro e insomma\\
		di sfiorare te prima\\
		di tornare a cercare me
	\end{verse}

\clearpage

\poemtitle{ix}

\vspace*{2cm}

	\begin{verse}
		io non so più densa\\
		intensità di senso\\
		non dignità maggiore\\
		che in questo tuo fermarmi
	\end{verse}

	\begin{verse}
		nel gesto che dice la lotta\\
		ti ritrai ti concedi\\
		cedendo al desiderio\\
		a lui soltanto cedi terreno\\
		non a me d'un tratto spettatore
	\end{verse}

	\begin{verse}
		vuoi forse proteggermi?\\
		sottrarmi al pericolo\\
		di quale disfatta\\
		di giudiziosi propositi?\\
		temo sia tardi alquanto\\
		per questo
	\end{verse}

	\begin{verse}
		ma lascia che mi lasci guidare\\
		dalla maestà del tuo gesto\\
		e in esso appaghi e non appaghi\\
		il mio desiderio
	\end{verse}

	\begin{verse}
		lascia che vi scopra la gioia\\
		di stare nelle tue mani\\
		compiutamente
	\end{verse}

\clearpage

\poemtitle{x}

\vspace*{2cm}

	\begin{otherlanguage}{french}
		\begin{verse}
			j'ai appris que mes jours sont\\
			pareils à une éponge de Menger\\
			au-dessus d'un wagon plat
		\end{verse}
	\end{otherlanguage}

\clearpage

\poemtitle{xi}

\vspace*{2cm}

	\begin{verse}
		fermo, mi dici\\
		col gesto che intendo\\
		oltre lo schermo\\
		più che con parole
	\end{verse}

	\begin{verse}
		che io non richiami\\
		parlando di voglia\\
		tra noi il fantasma\\
		che già ci possiede
	\end{verse}

	\begin{verse}
		che non si consumi\\
		quel desiderio\\
		andando per le bocche\\
		degli altri
	\end{verse}

	\begin{verse}
		quel desiderio \\
		che non si nasconde\\
		che presiede al tuo: fermo,\\
		più che a invito palese
	\end{verse}

\clearpage

\poemtitle{xii}

\vspace*{2cm}

	\begin{verse}
		ci si disperde più volentieri\\
		in superbo autarchico trionfo\\
		che nel concedersi a invito di donna o di uomo
	\end{verse}

	\begin{verse}
		ma disperdersi è compromettersi\\
		è assentire all'imperfezione di un consumo frettoloso\\
		e io non ricordo fu mai senza lesioni il mio piacere
	\end{verse}

\clearpage

\poemtitle{xiii}

\vspace*{2cm}

	\begin{verse}
		il mio sesso è uno sterpo\\
		spezzato e secco\\
		e il tuo è un nido di rovi\\
		ma uniamoci lo stesso\\
		forse non ci saranno spasimi\\
		forse ci incastreremo
	\end{verse}

\clearpage

\poemtitle{xiv}

\vspace*{2cm}

	\begin{verse}
		ogni giorno conosco il desiderio\\
		che la passante getta per gioco\\
		conio genuino in un berretto da clochard
	\end{verse}

	\begin{verse}
		dal che risponde al turgore di lei\\
		pienezza inequivocabile nella mia carne\\
		so che è uguale alla mia la sua argilla
	\end{verse}

	\begin{verse}
		anzi\\
		fu
	\end{verse}

	\begin{verse}
		ora\\
		è contorta da nasconderla, la mia,\\
		e da torcere lo sguardo che lei non ne rida
	\end{verse}

\clearpage

\poemtitle{xv}

\vspace*{2cm}

	\begin{verse}
		ti aggrumi\\
		nel mio dormiveglia affannato\\
		ti adorni di sostanza carnale\\
		di presenza che non so ignorare
	\end{verse}

	\begin{verse}
		sei scomoda\\
		le tue morbidezze celano spigoli\\
		ti sveli inghiottitrice di giorni\\
		e mi parli incongrua e stridente\\
		nella tana infeconda del letto
	\end{verse}

	\begin{verse}
		e mi abbaglia\\
		più chiara del sole la persuasione atroce\\
		che non ci saranno tempi di luce\\
		che è già parato il veleno
	\end{verse}

	\begin{verse}
		ma mi tradisce il risveglio\\
		e ora non so ritrovare il perché\\
		e mi lascio aggrovigliare\\
		nel tuo mistero
	\end{verse}

	\begin{verse}
		ancora
	\end{verse}

\clearpage

\poemtitle{xvi}

\vspace*{2cm}

	\begin{otherlanguage}{french}
		\begin{verse}
			ceci n'est que faire\\
			de l'archéologie\\
			je dit se branler\\
			par cœur
		\end{verse}

		\begin{verse}
			ce désir en revanche\\
			d'aujourd'hui\\
			je ne veux pas qu'il gagne\\
			la tchernaïa zemlia charnelle\\
			avant\\
			(pourvu qu'on franchît cela)\\
			qu'on nique
		\end{verse}
	\end{otherlanguage}

\clearpage

\poemtitle{xvii}

\vspace*{2cm}

	\begin{verse}
		fioritura di sangue\\
		in cristalli sapidi\\
		tesoro a milioni\\
		di tornesi pompeiani
	\end{verse}

	\begin{verse}
		colmarne la conca delle mani\\
		impiastricciarsi di sangue\\
		o mangiarseli uno a uno\\
		calibrando il capriccio?
	\end{verse}

	\begin{verse}
		è gioco da equilibrista\\
		non lasciarne cadere\\
		uno neppure dei grani\\
		di prosapia fenicia
	\end{verse}

	\begin{verse}
		a pena di sbucciare\\
		dai petali ardenti\\
		onde lo si sigillava\\
		un pezzetto di mondo
	\end{verse}

\clearpage

\poemtitle{xviii}

\vspace*{2cm}

	\begin{verse}
                guizzi improvvisi\\
                eccitazione e mollezze\\
                flaccida cruda carne si erge\\
                ricade
	\end{verse}

	\begin{verse}
                automatici gesti\\
                a placare così un vuoto che divora\\
                ma si denuncia da sé l’aporia
	\end{verse}

	\begin{verse}
                eppure non si saprebbe fermarsi prima d’aver concluso il lavoro
	\end{verse}

	\begin{verse}
                ed ha la meglio alla fine\\
                un magro corpo d’adolescente\\
                la sua lascivia tenera\\
                e la tenerezza d’un cedimento
	\end{verse}

	\begin{verse}
                effondersi quieti a onta delle carezze grevi
	\end{verse}

\clearpage

\poemtitle{xix}

\vspace*{2cm}

	\begin{verse}
                perché cerchiamo una presenza facile\\
                rassicurante\\
                un piatto lasciato in caldo\\
                un rifugio sicuro e parole amiche?
	\end{verse}

	\begin{verse}
                suvvia\\
                è ora di andare
	\end{verse}

	\begin{verse}
                ma non tale è la vita non può\\
                troppo ostentatamente prodiga\\
                offrirci un asilo sicuro e un letto sincero
	\end{verse}

	\begin{verse}
                troppo abbiamo indugiato\\
                bisogna andare\\
                al nostro amore consacreremo un pezzetto di ricordo\\
                al riparo dal sole
	\end{verse}

	\begin{verse}
                quanto a noi\\
                altre lusinghe fermeranno lo sguardo
	\end{verse}

\clearpage

\poemtitle{xx}

\vspace*{2cm}

	\begin{verse}
                certi rifugi scontrosi nella carne\\
                ti allettano con la promessa di un frammento\\
                di vita discreto e a buon mercato
	\end{verse}

	\begin{verse}
                ti lasci sedurre
	\end{verse}

	\begin{verse}
                ti lasci prendere a poco a poco\\
                nella ciclicità dorata e inesorabile\\
                che li sottende
	\end{verse}

	\begin{verse}
                ti lasci distogliere\\
                (con sollievo)\\
                dalla vita\\
                quella vera\\
                salata\\
                spigolosa
	\end{verse}

\clearpage

\poemtitle{xxi}

\vspace*{2cm}

	\begin{verse}
		perché ruba il fiato per un attimo\\
		il risucchio del vuoto\\
		dopo uno spasimo freddo\\
		dopo un abbaglio di pixel?\\
		ma giusto un attimo che tutto è ok\\
		si torna all'ordinata gestione\\
		scarichi ripuliti ipocritamente\\
		sorridenti alla vita
	\end{verse}

	\begin{verse}
		eppure anche di questo siamo fatti\\
		di voler strappare il senso\\
		all'incontro con l'altro\\
		senza incontrare nessuno\\
		di voler chiudere nel pugno il mistero\\
		di voler fare autarchica norma\\
		il bastare a noi stessi
	\end{verse}

	\begin{verse}
		eppure anche di questo siamo fatti\\
		di dimestichezza con le viltà della carne\\
		quando travolta irritata\\
		si ritira in buon ordine\\
		avendo concesso di sé al mondo\\
		non più che un infimo parsimonioso\\
		sporcarsi
	\end{verse}

\clearpage

\poemtitle{xxii}

\vspace*{2cm}

	\begin{verse}
		quei calici gemelli\\
		di vino misturato\\
		un po' stanchi dal peso\\
		io so a che somigliarli\\
		ma preferisco tacerlo\\
		non sta bene il parlare indecente
	\end{verse}

\clearpage

\poemtitle{xxiii}

\vspace*{2cm}

	\begin{verse}
		che malinconia dopo\\
		quando nella cenere si spengono i bengala\\
		quando rancidiscono le vivande appetitose
	\end{verse}

	\begin{verse}
		per dieci secondi il mondo è sporco
	\end{verse}

	\begin{verse}
		ma poi basta un recto in 16º per nettare\\
		la superficie immacolata del giorno
	\end{verse}

\clearpage

\poemtitle{xxiv}

\vspace*{2cm}

	\begin{verse}
		venisti qui per un gioco leggero\\
		per un giorno o due di carezze\\
		ma poi sei rimasta\\
		accoccolata tra le lenzuola\\
		come bestiola senza difese
	\end{verse}

	\begin{verse}
		non erano questi i taciti patti
	\end{verse}

\clearpage

\poemtitle{xxv}

\vspace*{2cm}

	\begin{verse}
		l'inferno lo si riconosce soltanto\\
		al percorrerlo aggrappati\\
		con le mani al sesso\\
		come a un plettro di lira\\
		o a un fuso intrecciato di spago
	\end{verse}

\clearpage

\poemtitle{xxvi}

\vspace*{2cm}

	\begin{verse}
		quando ci accorgiamo che ci serve\\
		un testimone per tutti i giorni\\
		lo troviamo in terre familiari\\
		compimento di un solco tracciato
	\end{verse}

	\begin{verse}
		poi però andiamo fuori a cercarci\\
		voglia di ribellione e paura\\
		e desiderio\\
		da vivere o da sentircene un attimo\\
		appena sfiorati
	\end{verse}

\clearpage

\poemtitle{xxvii}

\vspace*{2cm}

	\begin{verse}
		favo di miele\\
		gonfio appiccicoso\\
		dai riflessi del vetro\\
		come esce dal forno\\
		e poi è preso\\
		caldo sul ferro
	\end{verse}

	\begin{verse}
		il tuo volto ora è maschera\\
		antica di un dio\\
		di un atleta\\
		impetrato da morte\\
		lo sguardo appeso a un punto\\
		dietro i miei occhi
	\end{verse}

	\begin{verse}
		mi faccio serio e ti rendo\\
		il tributo che non si nega\\
		a chi cade sconfitto nell'arena\\
		prima di piegarti alla\\
		inevitabile resa
	\end{verse}

\clearpage

\poemtitle{xxviii}

\vspace*{2cm}

	\begin{verse}
		scivolo in cedimenti troppo noti\\
		perché mi facciano più tremare\\
		e preferisco acconsentire\\
		al veleno sottile delle giornate
	\end{verse}

	\begin{verse}
		meglio imparare dolcemente\\
		meglio lasciarsi cadere piano\\
		issare il vessillo della resa\\
		scendere a patti con tagliole incustodite
	\end{verse}

	\begin{verse}
		ma non so che pensare\\
		poiché so che così mi lascio scorrere tra le dita\\
		l'urgenza della mia scrittura\\
		e l'impossibilità di tacere
	\end{verse}

\clearpage

\poemtitle{xxix}

\vspace*{2cm}

	\begin{verse}
		che questo incontro\\
		non sperato non atteso\\
		ma soltanto\\
		studiatamente preparato\\
		trascorra in fretta\\
		come va tra la folla\\
		il commesso affannato\\
		alla stazione
	\end{verse}

\clearpage

\poemtitle{xxx}

\vspace*{2cm}

	\begin{verse}
		su raccogliamo le some\\
		di voglia e di colpa\\
		e concediamo asilo\\
		alla materia più fetida\\
		alla materia oscura del vivere\\
		quella che ai fortunati è permesso\\
		e agli ipocriti sdegnosi\\
		di gettare via tra le cose\\
		di cui non è bello parlare\\
		e che ci sia lieve il fardello\\
		per quanto possibile
	\end{verse}

\clearpage

\poemtitle{xxxi}

\vspace*{2cm}

	\begin{verse}
		continua a bruciare\\
		la fredda scambievole violenza che ci tiene insieme\\
		che a me strappa le ore minuto a minuto\\
		a me consapevole\\
		a me di marmo sanguinante\\
		come il giocatore incatenato al tavolo che lo spoglia
	\end{verse}

	\begin{verse}
		ma alla nuova se ci sarà io chiedo\\
		gli occhi senza sguardo\\
		i passi serrati a un binario\\
		che corre presso alla frontiera delle mie cose\\
		e una cura perentoria\\
		e giorni di angoscia q.b.
	\end{verse}

\clearpage

\poemtitle{xxxii}

\vspace*{2cm}

	\begin{verse}
		non ignaro del mio dovere\\
		mi costringo a tendere i sensi\\
		che ne sia appagato quello sguardo\\
		oltre la schiena salda al mio fianco
	\end{verse}

	\begin{verse}
		ma non so incoccare la freccia all'arco\\
		né so adattare il plettro alla lira\\
		perché il mio desiderio è andato\\
		dove non posso raggiungerlo
	\end{verse}

\clearpage

\poemtitle{xxxiii}

\vspace*{2cm}

	\begin{verse}
		cosa fummo\\
		se non due cammini\\
		intrecciati in terra straniera\\
		se non due ruscelli giovani\\
		geologie attraversate di nascosto alla luce\\
		poi gettati sopra la terra, insieme?
	\end{verse}

	\begin{verse}
		denunciati senza preavviso al sole\\
		stupore liquido nel colmo del giorno\\
		di specchiarci l'un l'altro\\
		nel corso gemello\\
		ribellione a scoprire\\
		ancora due solchi disgiunti
	\end{verse}

	\begin{verse}
		cosa fummo\\
		se non due uccelli giovani\\
		sporchi ancora di madre?\\
		ali da provare e una spezzata di volo\\
		a beccarci di gusto crudele\\
		a tagliarci l'un l'altro ogni guizzo di fuga
	\end{verse}

	\begin{verse}
		cosa fummo\\
		se non due ragazzi giovani\\
		il tempo scoperto d'un tratto\\
		mesi da correre d'un fiato\\
		il mondo rifatto nuovo e nostro?
	\end{verse}
	
	\begin{verse}
		e non volerne più sapere\\
		o non ancora\\
		di anni diversi e più vili\\
		da soli
	\end{verse}

\clearpage

\poemtitle{xxxiv}

\vspace*{2cm}

	\begin{verse}
		presenza che non so\\
		se tu sia più pervasiva o più accogliente\\
		corpo chimerico opalescente traslucido\\
		da alterarne il tempo e lo spazio\\
		come massiva stella\\
		lasciala intatta la geometria del mio mondo\\
		ché ho bisogno di una percezione spigolosa\\
		non di te molle e accogliente
	\end{verse}

\clearpage

\poemtitle{xxxv}

\vspace*{2cm}

	\begin{verse}
		dove la pienezza mischiata\\
		di sospensione angosciosa\\
		dove la pienezza antica\\
		odorosa di seme non estraneo?
	\end{verse}

	\begin{verse}
		ora l’angoscia insistente\\
		bordone alla vita di un uomo\\
		grumo che volta a volta\\
		si contrae a ogni effusione
	\end{verse}

	\begin{verse}
		nulla cambia a questo seme\\
		che a sollecitarlo\\
		o e ad accoglierlo\\
		sia carne altrui
	\end{verse}

\clearpage

\poemtitle{xxxvi}

\vspace*{2cm}

	\begin{verse}
		che siamo ora, mia diletta?\\
		anche se fuggono i giorni\\
		e per scherno ha violato il tempo\\
		i bei corpi (orgoglio di giovinezza)\\
		non altro saprei trovarmi intorno\\
		che valga i tuoi seni appassiti
	\end{verse}

\clearpage

\poemtitle{xxxvii}

\vspace*{2cm}

	\begin{otherlanguage}{french}
		\begin{verse}
			dès le matin décharger vite\\
			avant de s'habiller pour la journée
		\end{verse}
	\end{otherlanguage}
