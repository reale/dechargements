\begin{volumetitlepage}
	\volumetitle{Geologia}
	\volumeheader{Geologia}
\end{volumetitlepage}

\poemtitle{i}

\begin{poem}
	\begin{stanza}
		a Meleagro\verseline
		il figlio d'Oineo\verseline
		la madre gli mise la vita nel fuoco\verseline
		bastò un gesto della mano
	\end{stanza}

	\begin{stanza}
		del padre invece non raccontano molto
	\end{stanza}
\end{poem}

\clearpage

\poemtitle{ii}

\begin{artItem}
	Joel-Peter Witkin, \begin{otherlanguage}{english}%
		Invention of milk%
	\end{otherlanguage}
\end{artItem}

\begin{poem}
	\begin{stanza}
		tutti quanti incontriamo dopo —\verseline
		nessuno di loro ci procura la fine\verseline
		ché sono in genere soltanto compagni\verseline
		o alla peggio lupi maldestri
	\end{stanza}

	\begin{stanza}
		e non danno\verseline
		— aspra talvolta —\verseline
		che vita
	\end{stanza}

	\begin{stanza}
		invece il modo e il tempo della fine\verseline
		segnati già nel modo di toccare\verseline
		o nel primitivo sguardo\verseline
		o nel negarsi saldo di questo\verseline
		li decide la madre
	\end{stanza}
\end{poem}

\clearpage

\poemtitle{iii}

\begin{artItem}
	Arturo Martini, La madre folle
\end{artItem}

\begin{poem}
	\begin{stanza}
		gli occhi dove non fu mai remissione\verseline
		non so se li nascondesti da noi\verseline
		o soltanto li negavi al mondo\verseline
		ma noi noi li cercavamo da prima ancora\verseline
		che ci gettasti sulle spiagge di luce\verseline
		né ancora abbiamo smesso
	\end{stanza}
\end{poem}

\clearpage

\poemtitle{iv}

\begin{artItem}
	Victor Vasarely, \begin{otherlanguage}{english}%
		Catch%
	\end{otherlanguage}
\end{artItem}

\begin{poem}
	\begin{stanza}
		mostro i denti arretrando\verseline
		a un approccio di madre\verseline
		a una carezza tardiva\verseline
		come ad agguato di belva\verseline
		gelosa del sangue
	\end{stanza}

	\begin{stanza}
		non sono che scolta ignara\verseline
		di sosta denti serrati\verseline
		allarme di lupo stretto da cani\verseline
		sguardo fisso in sguardo\verseline
		primario impervio di madre
	\end{stanza}
\end{poem}

\clearpage

\poemtitle{v}

\begin{artItem}
	Giovanni Anselmo, senza titolo, granito, carne, filo di rame
\end{artItem}

\begin{poem}
	\begin{stanza}
		della terra\verseline
		è impura — mi dissero —\verseline
		tu non mescolartici\verseline
		e fu tra le beffe\verseline
		non la meno atroce
	\end{stanza}

	\begin{stanza}
		della terra\verseline
		solo poi mi feci esperto\verseline
		la sanno il mio dorso il petto\verseline
		le braccia che la tengono ferma
	\end{stanza}

	\begin{stanza}
		della terra\verseline
		so cosa amo\verseline
		non l'accoglienza di argille di sabbia\verseline
		ma il granito aspro\verseline
		senza cedimenti\verseline
		indifferente
	\end{stanza}
\end{poem}

\clearpage

\poemtitle{vi}

\begin{poem}
	\begin{stanza}
		né macigno da scagliare\verseline
		per detronizzarti\verseline
		né per farmene\verseline
		difesa da te\verseline
		estrema e fragile
	\end{stanza}

	\begin{stanza}
		era per sederci\verseline
		insieme e per parlarci\verseline
		e per vederci\verseline
		come avremmo dovuto
	\end{stanza}

	\begin{stanza}
		non scoprirti a scegliermi\verseline
		il peggiore tra i nemici\verseline
		il più vile\verseline
		non per sapermi\verseline
		capace di violarti\verseline
		che neppure ti arrivavo\verseline
		con la testa al petto
	\end{stanza}
\end{poem}

\clearpage

\poemtitle{vii}

\begin{artItem}
	George Tooker, \begin{otherlanguage}{english}%
		Voice I%
	\end{otherlanguage}
\end{artItem}

\begin{poem}
	\begin{stanza}
		nel momento di far quadra dei conti\verseline
		so che non si colpisce per lacerare\verseline
		ma per scoprire la carne\verseline
		per sapere la geologia dei legami
	\end{stanza}

	\begin{stanza}
		che non siamo di sostanza nemica\verseline
		lo scopersi nelle tue spalle\verseline
		lo riconobbi al gesto con cui ti schermivi\verseline
		e ci fa vicini vergogna e ribellione\verseline
		e dispersa sapienza d’amore
	\end{stanza}

	\begin{stanza}
		ora nelle tue mani non trovo più\verseline
		angoscia — di quella non voglio parlare —\verseline
		ma voto d’intelletto reciproco\verseline
		e lo so il segreto — che nascondi\verseline
		così bene — ma t'ho visto —
	\end{stanza}

	\begin{stanza}
		voglio che sia scelta\verseline
		il nostro primo incontro
	\end{stanza}
\end{poem}

\clearpage

\poemtitle{viii}

\begin{poem}
	\begin{stanza}
		da che li seppi ignari\verseline
		di umanità corrente\verseline
		li volli incolpevoli automi\verseline
		né ho poi più lasciato\verseline
		di cercarli altrove
	\end{stanza}
\end{poem}
