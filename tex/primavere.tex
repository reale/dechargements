\begin{volumetitlepage}
	\volumetitle{Primavere}
	\volumeheader{Primavere}
\end{volumetitlepage}

\poemtitle{i}

\begin{artItem}
	Gaetano Previati, Paesaggio
\end{artItem}

\begin{poem}
	\begin{stanza}
		schiarisce il cielo\verseline
		un attimo prima fingeva inverno
	\end{stanza}
\end{poem}

\clearpage

\poemtitle{ii}

\begin{poem}
	\begin{stanza}
		sonno mescolato di voci\verseline
		di luce\verseline
		di pioggia impròvvida di marzo
	\end{stanza}

	\begin{stanza}
		corsa a mezz’aria sui binari\verseline
		realtà o inganno dei sensi?
	\end{stanza}

	\begin{stanza}
		poi, giunto, ho scelto\verseline
		l’oblio delle coltri
	\end{stanza}
\end{poem}

\clearpage

\poemtitle{iii}

\begin{poem}
	\begin{otherlanguage}{french}
		\begin{stanza}
			la haine et l’ivresse surveillent\verseline
			l’enfantement d’un jour
		\end{stanza}
	\end{otherlanguage}
\end{poem}

\clearpage

\poemtitle{iv}

\begin{poem}
	\begin{stanza}
		si cammina come sospesi\verseline
		in un pulviscolo che si fa scuro\verseline
		e agguaglia terra e cielo
	\end{stanza}

	\begin{stanza}
		ci si aggrappa a questi pezzi\verseline
		di luce che invadono l’aria
	\end{stanza}
\end{poem}

\clearpage

\poemtitle{v}

\begin{poem}
	\begin{stanza}
		il cielo è questo grembo immenso\verseline
		avvolgente e intoccabile\verseline
		sulle sue
	\end{stanza}

	\begin{stanza}
		con incongruità ben nota\verseline
		mi infiamma la primavera voglia di tana
	\end{stanza}
\end{poem}

\clearpage

\poemtitle{vi}

\begin{poem}
	\begin{stanza}
		sere di marzo che già si slargano\verseline
		il cielo conca di barbagli\verseline
		una presenza-assenza\verseline
		di presentiti turbamenti marini
	\end{stanza}
\end{poem}

\clearpage

\poemtitle{vii}

\begin{poem}
	\begin{stanza}
                primavera è anche questa\verseline
                col cielo pieno di pioggia\verseline
                perché si sente la lotta\verseline
                ma si sa la certezza del sereno
	\end{stanza}
\end{poem}

\clearpage

\poemtitle{viii}

\begin{artItem}
	Hermann Nitsch, \begin{otherlanguage}{german}%
		Das 6-Tage-Spiel%
	\end{otherlanguage}
\end{artItem}

\begin{poem}
	\begin{stanza}
		ancora aspetto un apollo\verseline
		che venga a togliermi di dosso\verseline
		la pelle che mi germogliò\verseline
		l'anno passato
	\end{stanza}
\end{poem}

\clearpage

\poemtitle{ix}

\begin{artItem}
	Marc Chagall, \begin{otherlanguage}{russian}%
		Посвящение Гоголю%
	\end{otherlanguage}
\end{artItem}

\begin{poem}
	\begin{stanza}
		questa primavera mi è scoppiata dentro\verseline
		come di consolidata prammatica\verseline
		ma poi che fine ha fatto la voglia
	\end{stanza}

	\begin{stanza}
		(che avevo)
	\end{stanza}

	\begin{stanza}
		di non lasciarne scorrere a vuoto\verseline
		i giorni gonfi di luce?
	\end{stanza}
\end{poem}

\clearpage

\poemtitle{x}

\begin{poem}
	\begin{stanza}
		vento novello di maggio\verseline
		mi trascina fuori di casa\verseline
		mi vuole ebbro degli umori\verseline
		di un Mediterraneo ridesto...
	\end{stanza}
\end{poem}
