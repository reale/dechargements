\begin{volumetitlepage}
	\volumetitle{A gravame}
	\volumeheader{A gravame}
	\bigskip\bigskip\bigskip
	\volumeepigraph{
		\begin{verse}
			\begin{otherlanguage}{french}
				Ço dit Ysolt : Jol sai pur veir.\\
				Sachez que le sigle est tut neir.
			\end{otherlanguage}
		\end{verse}
	}
	\volumeattribution{Tumas de Britanje}
\end{volumetitlepage}

\poemtitle{i}

\begin{artItem}
	Jeremy Mann, \begin{otherlanguage}{english}%
		Times Square Lights%
	\end{otherlanguage}
\end{artItem}

\begin{poem}
	\begin{otherlanguage}{english}
		\begin{stanza}
			we hear the mournful knells
		\end{stanza}

		\begin{stanza}
			the glorious glare of day\verseline
			quenched with gloom\verseline
			hath yielded to night
		\end{stanza}

		\begin{stanza}
			the luscious scent of blossoms\verseline
			is fraught with rue\verseline
			with bitter reproach
		\end{stanza}

		\begin{stanza}
			and lo! from afar we spy\verseline
			the gleamy lights\verseline
			and the dismal attire\verseline
			of the torch-bearers
		\end{stanza}
\end{otherlanguage}
\end{poem}

\clearpage

\poemtitle{ii}

\begin{artItem}
	Arsenij Nikiforovič Semënov \begin{otherlanguage}{russian}%
		Вид на Смольный собор в Ленинграде%
	\end{otherlanguage}
\end{artItem}

\begin{poem}
	\begin{stanza}
		alle luci dell'alba, alla strada già chiara\verseline
		alle torce (inutili) tra le mani\verseline
		al voltare della cantonata\verseline
		con la via che slarga verso il mare\verseline
		al troppo\verseline
		lezzo di primavera che comincia\verseline
		(finalmente)\verseline
		e ai tuoi seni barbari, per ultimi
	\end{stanza}
\end{poem}

\clearpage

\poemtitle{iii}

\begin{artItem}
	Sabin Bălaşa, \begin{otherlanguage}{romanian}%
		Nunta cosmică%
	\end{otherlanguage}
\end{artItem}

\begin{poem}
	\begin{stanza}
		il confine tra veglia chiacchierina\verseline
		e sonno inconsapevole\verseline
		l’ho superato stanotte mille volte\verseline
		come un fiume basso da passare a guado
		in un senso prima e poi nell’altro
	\end{stanza}

	\begin{stanza}
		se ai miei occhi ogni frontiera\verseline
		non è mai netta né indiscutibile\verseline
		ma frastagliata e gonfia e terra ampia\verseline
		è alle notti agitate è alla contesa aperta\verseline
		di sonno e veglia che ne son debitore
	\end{stanza}
\end{poem}

\clearpage

\poemtitle{iv}

\begin{artItem}
	Giacomo Balla, Mercurio transita davanti al sole
\end{artItem}

\begin{poem}
	\begin{stanza}
		ancora mi seduce la sostanza oscura\verseline
		e calda delle notti sicché corro\verseline
		a rifugiarmi tra le coltri\verseline
		preso da voglia di tana
	\end{stanza}

	\begin{stanza}
		e che siano serrate a doppia mandata\verseline
		le porte che non vi s'insinui\verseline
		agonia viscida d'inverno
	\end{stanza}

	\begin{stanza}
		e non è più il tempo che mi lascio\verseline
		strappare al sonno\verseline
		da voci trascinate nel buio\verseline
		ma che non lo vorrei ancora una volta\verseline
		non chiedetemi giuramento
	\end{stanza}
\end{poem}

\clearpage

\poemtitle{v}

\begin{poem}
	\begin{stanza}
		cari muri d’edera e vecchi lampioni\verseline
		quante volte mi foste quinta\verseline
		a un vagare interminato\verseline
		compagni nell’imbarazzo del mattino\verseline
		scorta di strada senza fine
	\end{stanza}

	\begin{stanza}
		ma non lasciate amici\verseline
		che io sia solo stanotte\verseline
		negli angoli gremiti di rottami\verseline
		si appiatta il corteo dei giorni\verseline
		l’affanno mi toglie le forze\verseline
		il non saper dire\verseline
		ancora\verseline
		sono nascosti\verseline
		laggiù
	\end{stanza}

	\begin{stanza}
		e ieri ho sentito il calore degli amici\verseline
		per la prima volta una comunità\verseline
		inoffensiva\verseline
		non lasciarsi squarciare il petto\verseline
		dalla durezza degli sguardi
	\end{stanza}
\end{poem}

\clearpage

\poemtitle{vi}

\begin{poem}
	\begin{stanza}
		non è ancora allentata\verseline
		la morsa chiusa dell'inverno\verseline
		ma lo sarà presto
	\end{stanza}

	\begin{stanza}
		come un esercito schierato\verseline
		come le schiere rosse dell'armata fuori dai reticolati\verseline
		così è la primavera tutt'intorno alle mura\verseline
		e manda lontano i canti di guerra e le grida
	\end{stanza}

	\begin{stanza}
		io non sbircio dalle finestre\verseline
		ma spingo un braccio oltre gli interstizi della tana
	\end{stanza}
\end{poem}

\clearpage

\poemtitle{vii}

\begin{artItem}
	Giulio Turcato, Superficie lunare
\end{artItem}

\begin{poem}
	\begin{stanza}
		nient’altro che\verseline
		un contorcersi di nero nel nero\verseline
		però un contorcersi regale\verseline
		e artificiali lucciole che non si contavano\verseline
		e la solita notte incorente\verseline
		la solita notte in problematico equilibrio\verseline
		la solita notte che non dà pace
	\end{stanza}
\end{poem}

\clearpage

\poemtitle{viii}

\begin{artItem}
	Alberto Burri, Grande cretto nero
\end{artItem}

\begin{poem}
	\begin{stanza}
		si strappa il velo del cielo\verseline
		si strappa ad oriente la tregua della notte\verseline
		troppo tardi per negarsi alla crudeltà primaverile\verseline
		troppo tardi per aggrapparsi all'inverno che dilegua\verseline
		e il giorno dilaga tra le pietre\verseline
		e per le vie di giovinezza antica\verseline
		le membra stracche dal carico inutile e lungo\verseline
		si riposano\verseline
		e non si risparmiano gli schiaffi sulle spalle dei compagni\verseline
		che ormai la fatica è compiuta\verseline
		e poi a stare tutti insieme si sente meno\verseline
		questo giorno che comincia stupito e senza ombre\verseline
		fiaccole opache e inutili\verseline
		estinte a una a una\verseline
		con gesto di carnefice
	\end{stanza}
\end{poem}

\clearpage

\poemtitle{ix}

\begin{artItem}
	Mimmo Paladino, Dormiente
\end{artItem}

\begin{poem}
	\begin{stanza}
		giorno pieno
	\end{stanza}

	\begin{stanza}
		per la finestra spalancata\verseline
		si versa dentro alla stanza\verseline
		il cielo di nuvolaglia
	\end{stanza}

	\begin{stanza}
		un brandello di sonno ancora\verseline
		strappato agli uffici mattinali
	\end{stanza}

	\begin{stanza}
		è il sogno del padre\verseline
		finalmente remoto\verseline
		(scoppio d’odio o di passione tardiva?)\verseline
		l’estrema difesa
	\end{stanza}

	\begin{stanza}
		risvegliarsi poi alla ferraglia\verseline
		del tram\verseline
		che riempie la strada
	\end{stanza}
\end{poem}

\clearpage

\poemtitle{x}

\begin{poem}
	\begin{stanza}
		non è ancora\verseline
		o non è più il tempo\verseline
		di prestare orecchio\verseline
		a un remoto dolore carnale\verseline
		eppure\verseline
		troppo di me c'è rimasto incastrato\verseline
		onde il continuo riandare
	\end{stanza}
\end{poem}

\clearpage

\poemtitle{xi}

\begin{poem}
	\begin{stanza}
		sgorgare di voci\verseline
		rorida messe\verseline
		di spighe feconde
	\end{stanza}

	\begin{stanza}
		sensi tesi\verseline
		incerta sapienza\verseline
		disperazione carnale
	\end{stanza}

	\begin{stanza}
		memorie di una\verseline
		perduta pubertà\verseline
		passione voluta gridare
	\end{stanza}

	\begin{stanza}
		ma soffocata nella carne\verseline
		un passo prima di primavera
	\end{stanza}
\end{poem}

\clearpage

\poemtitle{xii}

\begin{artItem}
	Lyonel Feininger, \begin{otherlanguage}{english}%
		Sails%
	\end{otherlanguage}
\end{artItem}

\begin{poem}
	\begin{stanza}
		vessillo esposto e spregiato\verseline
		panno mondato di macchie\verseline
		lenzuolo tradito di nozze
	\end{stanza}

	\begin{stanza}
		schiocchi appiccicosi\verseline
		come sofferenza dolce\verseline
		vessillo teso nell'aria\verseline
		tesa essa pure di freddo\verseline
		che va via
	\end{stanza}

	\begin{stanza}
		sera lontana di primavera\verseline
		lontana sera di strazio\verseline
		di vento e di sensi\verseline
		di dolorosa nascita\verseline
		di remota venuta al mondo\verseline
		scaglia di notte più densa\verseline
		straccio tessuto troppo in fretta\verseline
		telo riempito di paure\verseline
		cortina spregiata
	\end{stanza}

	\begin{stanza}
		messe di vento\verseline
		facies di antiche nozze\verseline
		forma di geometria non dedotta\verseline
		incalzare non placato\verseline
		non ancora
	\end{stanza}
\end{poem}

\clearpage

\poemtitle{xiii}

\begin{poem}
	\begin{stanza}
		c’è dentro più di durezza\verseline
		negli artigli d'astore dell'inverno\verseline
		aggrappati alle carni\verseline
		o nel dilatarsi del buio\verseline
		di questa notte?
	\end{stanza}

	\begin{stanza}
		s'immagina lontana l'alba\verseline
		ma non è\verseline
		un gallo riempie la cavità della notte\verseline
		di un gracchiare fuori tempo\verseline
		forse per la troppo incongrua\verseline
		dilatazione
	\end{stanza}

	\begin{stanza}
		è fresco
	\end{stanza}

	\begin{stanza}
		il lenzuolo un cencio lasciato in un angolo\verseline
		il gallo neppure fa più paura\verseline
		sta lì muto o stride ogni tanto
	\end{stanza}
\end{poem}
