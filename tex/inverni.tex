\begin{volumetitlepage}
	\volumetitle{Inverni}
	\volumeheader{Inverni}
	\bigskip\bigskip\bigskip
	\volumeepigraph{
		\begin{verse}
			\begin{otherlanguage}{greek}
				τὸ τέλος ὁ χρόνος ἀπαιτεῖ
			\end{otherlanguage}
		\end{verse}
	}
	\volumeattribution{Epitaffio di Sicilo}
\end{volumetitlepage}

\poemtitle{i}

\begin{poem}
	\begin{stanza}
		virgulto tenero d’autunno\verseline
		novembre nato in fretta\verseline
		non più che un presentimento\verseline
		di braci e castagne
	\end{stanza}

	\begin{stanza}
		la forza adorata dell’estate\verseline
		corse via in fretta\verseline
		ora dita gelide tentano gli scuri
	\end{stanza}
\end{poem}

\clearpage

\poemtitle{ii}

\begin{poem}
	\begin{stanza}
		solo gesto di ribellione\verseline
		offrirsi alle raffiche nudi\verseline
		respirare aria secca tagliente\verseline
		lasciarsi scovare qualcosa\verseline
		da qualche parte
	\end{stanza}
\end{poem}

\clearpage

\poemtitle{iii}

\begin{artItem}
	Antonín Slavíček, \begin{otherlanguage}{czech}%
		Rybník%
	\end{otherlanguage}
\end{artItem}

\begin{poem}
	\begin{stanza}
		questo novembre\verseline
		ci è scappato tra le dita\verseline
		come un cucciolo irrequieto
	\end{stanza}
\end{poem}

\clearpage

\poemtitle{iv}

\begin{poem}
	\begin{stanza}
		la stagione già inclina\verseline
		al tempo che è più matura\verseline
		più piena la sua luce\verseline
		ma c’è rimasto l’inverno\verseline
		impigliato tra i gesti
	\end{stanza}
\end{poem}

\clearpage

\poemtitle{v}

\begin{poem}
	\begin{stanza}
		che la notte di gennaio\verseline
		cuore di nebbia e strida di civette\verseline
		non sia inospite a pensieri meridiani\verseline
		— che sono disumani\verseline
		— come sappiamo
	\end{stanza}
\end{poem}
